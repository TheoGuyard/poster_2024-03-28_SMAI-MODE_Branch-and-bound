\documentclass[final]{beamer}

\usepackage[scale=1.24]{beamerposter}
\usepackage{tikz}
\usepackage{pifont}
\usetikzlibrary{matrix}
\usetikzlibrary{positioning}
\usetikzlibrary{er}

\usetheme{confposter}

\setbeamercolor{block title}{fg=dblue,bg=white}
\setbeamercolor{block body}{fg=black,bg=white}
\setbeamercolor{block alerted title}{fg=white,bg=dblue!70}
\setbeamercolor{block alerted body}{fg=black,bg=dblue!10}
\setbeamercolor{item}{fg=nblue}
\setbeamercolor{item projected}{fg=white,bg=nblue}

\newlength{\sepwid}
\newlength{\onecolwid}
\newlength{\twocolwid}
\newlength{\threecolwid}
\setlength{\paperwidth}{46.8in} % A0 width: 46.8in
\setlength{\paperheight}{33.1in} % A0 height: 33.1in
\setlength{\sepwid}{0.024\paperwidth} % Separation width (white space) between columns
\setlength{\onecolwid}{0.22\paperwidth} % Width of one column
\setlength{\twocolwid}{0.464\paperwidth} % Width of two columns
\setlength{\threecolwid}{0.708\paperwidth} % Width of three columns
\setlength{\topmargin}{-1.5in} % Reduce the top margin size

\usepackage{kmath}
% Tikz
\usetikzlibrary{calc}
\usetikzlibrary{mindmap,trees,shapes,arrows,backgrounds,topaths}
\usetikzlibrary{decorations.pathmorphing, shapes.geometric}

% Text
\usepackage{enumitem}
\usepackage{ulem}
\usepackage{pifont}

% Maths
\usepackage{amsmath}
\usepackage{amsfonts}
\usepackage{amsthm}
\usepackage{amsopn}

% Plots
\usepackage{pgfplots}
\usepgfplotslibrary{groupplots}

% Tables
\usepackage{booktabs}
\usepackage{array}
\newcolumntype{L}{>$l<$}
\arraycolsep=1.4pt
\setlength{\tabcolsep}{3pt}

% Algos
\usepackage[ruled]{algorithm2e}

% Pgfplot
\pgfplotsset{
    legend image code/.code={
        \draw[mark repeat=2,mark phase=2] plot coordinates {
            (0cm,0cm)
            (0.25cm,0cm)
            (0.25cm,0cm)
        };
    }
}
% Objective values and functions
\newcommand{\pobj}{p}
\newcommand{\robj}{r}
\newcommand{\dobj}{d}

% Variables
\newcommand{\pvletter}{x}
\newcommand{\wvletter}{w}
\newcommand{\dvletter}{u}
\newcommand{\vvletter}{v}
\newcommand{\bvletter}{z}
\newcommand{\pv}{\mathbf{\pvletter}}
\newcommand{\wv}{\mathbf{\wvletter}}
\newcommand{\dv}{\mathbf{\dvletter}}
\newcommand{\vv}{\mathbf{\vvletter}}
\newcommand{\bv}{\mathbf{\bvletter}}
\newcommand{\pvi}[1]{\pvletter_{#1}}
\newcommand{\wvi}[1]{\wvletter_{#1}}
\newcommand{\dvi}[1]{\dvletter_{#1}}
\newcommand{\vvi}[1]{\vvletter_{#1}}
\newcommand{\bvi}[1]{\bvletter_{#1}}

% Problem data
\newcommand{\pdim}{n}
\newcommand{\ddim}{m}
\newcommand{\dic}{\mathbf{A}}
\newcommand{\dici}[1]{\mathbf{a}_{#1}}
\newcommand{\dicii}[1]{a_{#1}}
\newcommand{\obs}{\mathbf{y}}
\newcommand{\obsi}[1]{y_{#1}}
\newcommand{\reg}{\lambda}
\newcommand{\groundtruth}{\pv^{\dagger}}
\newcommand{\lfunc}{f}
\newcommand{\pfunc}{h}
\newcommand{\rfunc}{g}
\newcommand{\dfunc}{D}
\newcommand{\relaxrfunc}{\tilde{g}}
\newcommand{\relaxpfunc}{\tilde{h}}
\newcommand{\bigM}{M}
\newcommand{\regtwo}{\alpha}
\newcommand{\rslope}{\tau}
\newcommand{\rlimit}{\mu}
\newcommand{\noise}{\boldsymbol{\epsilon}}

% Indices
\newcommand{\idxentry}{i}

% BnB
\newcommand{\pset}{\mathcal{X}}
\newcommand{\setidx}{\mathcal{S}}
\newcommand{\setzero}{\setidx_0}
\newcommand{\setone}{\setidx_1}
\newcommand{\setnone}{\setidx_\bullet}
\newcommand{\nodeSymb}{\nu}
\newcommand{\node}[1]{#1^{\nodeSymb}}

% Screening
\newcommand{\saferegion}{\mathcal{R}}
\newcommand{\safesphere}{\mathcal{S}}
\newcommand{\spherecenter}{\mathbf{c}}
\newcommand{\sphereradius}{r}

% Peeling
\newcommand{\bigL}{\boldsymbol{\alpha}}
\newcommand{\bigU}{\boldsymbol{\beta}}
\newcommand{\bigLi}[1]{\alpha_{#1}}
\newcommand{\bigUi}[1]{\beta_{#1}}

% Math operators
\DeclareMathOperator{\argmax}{argmax}
\DeclareMathOperator{\argmin}{argmin}
\DeclareMathOperator{\biconjugate}{biconj}
\DeclareMathOperator{\card}{card}
\DeclareMathOperator{\complset}{cmpl}
\DeclareMathOperator{\convex}{cvx}
\DeclareMathOperator{\diam}{diam}
\DeclareMathOperator{\dom}{dom}
\DeclareMathOperator{\interior}{int}
\DeclareMathOperator{\prox}{prox}
\DeclareMathOperator{\rank}{rank}
\DeclareMathOperator{\sign}{sign}


% Math misc
\newcommand{\1}{\mathbf{1}}
\newcommand{\0}{\mathbf{0}}
\newcommand{\abs}[1]{|#1|}
\newcommand{\biconj}[1]{#1^{**}}
\newcommand{\bigO}{\mathcal{O}}
\newcommand{\conj}[1]{#1^{*}}
\newcommand{\icvx}{\eta}
\newcommand{\intervint}[2]{[#1,#2]}
\newcommand{\iter}[2]{#1^{#2}}
\newcommand{\norm}[2]{\|#1\|_#2}
\newcommand{\opt}[1]{#1^{\star}}
\newcommand{\pospart}[1]{[#1]_+}
\newcommand{\separable}[2]{#1_{#2}}
\newcommand{\subdiff}{\partial}
\newcommand{\transp}[1]{#1^{\mathrm{T}}}

% Edition macros
\newcommand{\AddTodo}[1]{\textcolor{red}{[#1]}}
\usepackage[pscoord]{eso-pic}
\newcommand{\placetextbox}[3]{
  \setbox0=\hbox{#3}
  \AddToShipoutPictureFG*{
    \put(\LenToUnit{#1\paperwidth},\LenToUnit{#2\paperheight}){\vtop{{\null}\makebox[0pt][c]{#3}}}
  }
}
\usepackage[ruled]{algorithm2e}
\usepackage{enumitem}

\newcommand{\emphone}[1]{\textbf{\color{norange}#1}}

\newacronym{bnb}{BnB}{Branch-and-Bound}
\newacronym{mip}{MIP}{Mixed-Integer Program}

\title{Unifying Branch-and-Bound methods for $\mathbf{\ell}_{\0}$-penalized problems}

\author{Théo Guyard${}^{\star}$, Cédric Herzet${}^{\dagger}$, Clément Elvira${}^{\ddagger}$, Ay\c{s}e-Nur Arslan${}^{\diamond}$}

\institute{${}^{\star}$Inria and Insa Rennes, ${}^{\dagger}$Ensai, ${}^{\ddagger}$CentraleSupélec, ${}^{\diamond}$Inria Bordeaux}

\begin{document}


\addtobeamertemplate{block end}{}{\vspace*{2ex}}
\addtobeamertemplate{block alerted end}{}{\vspace*{2ex}}

\setlength{\belowcaptionskip}{2ex}
\setlength\belowdisplayshortskip{2ex}

\begin{frame}[t]

\begin{columns}[t]

\begin{column}{\sepwid}\end{column}

\begin{column}{\onecolwid}

    \setbeamercolor{block alerted title}{fg=white,bg=norange}
    \setbeamercolor{block alerted body}{fg=black,bg=white}
    \begin{alertblock}{In short}
        \centering
        Provide necessary ingredients to implement \emphone{generic} and \emphone{efficient} $\ell_0$-problem solvers.
    \end{alertblock}

    \begin{block}{}
        \setbeamercolor{block alerted title}{fg=white,bg=dblue!70}
        \setbeamercolor{block alerted body}{fg=black,bg=dblue!10}
        \begin{alertblock}{$\boldsymbol{\ell}_{\0}$-problem}
            \centering
            \begin{tikzpicture}
                \node (problem) at (current page.north) {$\opt{\pobj} = \textstyle\min_{\pv \in \kR^{\pdim}} \lossfunc(\dic\pv) + \reg\norm{\pv}{0} + \pertfunc(\pv)$};
                %
                \node[font=\small] (mip) at ($(problem.south)+(-6,-2)$) {\textbf{Generic solvers}};
                \node[font=\small] (mip-speed) at ($(mip.south)+(0,-0.75)$)  {\textcolor{OrangeRed}{\ding{55} Slow}};
                \node[font=\small] (mip-flexible) at ($(mip.south)+(0,-2)$)  {\textcolor{ForestGreen}{\ding{51} Flexible w.r.t $\lossfunc/\pertfunc$}};
                %
                \node[font=\small] (bnb) at ($(problem.south)+(6,-2)$) {\textbf{Specialized solvers}};
                \node[font=\small] (bnb-speed) at ($(bnb.south)+(0,-0.75)$)  {\textcolor{ForestGreen}{\ding{51} Fast}};
                \node[font=\small] (bnb-flexible) at ($(bnb.south)+(0,-2)$)  {\textcolor{OrangeRed}{\ding{55} Only some $\lossfunc/\pertfunc$}};
                %
                \node[font=\small] (contrib) at ($(problem.south)+(0,-8)$) {\textbf{Proposed solver}};
                \node[font=\small] (contrib-speed) at ($(contrib.south)+(0,-0.75)$)  {\textcolor{ForestGreen}{\ding{51} Fast}};
                \node[font=\small] (contrib-flexible) at ($(contrib.south)+(0,-2)$) {\textcolor{ForestGreen}{\ding{51} Flexible w.r.t $\lossfunc/\pertfunc$}};
            \end{tikzpicture}
        \end{alertblock}
        ~\\
        \textbf{Working hypotheses}
        \begin{itemize}[label=$\bullet \ $,leftmargin=2em]
            \item $\lossfunc$ simple to work with (convex, smooth, ...)
            \item $\pertfunc$ proper, closed, separable, cont. at $\pv=\0$
            \item[\textcolor{lightgray}{$\bullet \ $}] \textcolor{lightgray}{$\pertfunc$ even, convex, coercive, $\pertfunc \geq \pertfunc(\0)=0$}
        \end{itemize}
        ~\\
        \textbf{Versatile toolbox}
        \begin{figure}
            \begin{tikzpicture}
                \node (origin) at ($(current page.north)+(0,0)$) {};
                %
                \node[draw,line width=5pt,text width=0.35\linewidth,align=center,rounded corners=0.5cm,inner sep=24] (lossfunc) at ($(origin)+(-6,-6.5)$) {\textbf{User-defined $\lossfunc$} \\
                \texttt{func\_value} \\
                \texttt{conj\_value} \\
                \texttt{prox\_value} \\
                \texttt{subd\_value}};
                %
                \node[draw,line width=5pt,text width=0.35\linewidth,align=center,rounded corners=0.5cm,inner sep=24] (pertfunc) at ($(origin)+(6,-6.5)$) {\textbf{User-defined $\pertfunc$} \\
                \texttt{func\_value} \\
                \texttt{conj\_value} \\
                \texttt{prox\_value} \\
                \texttt{subd\_value}};
                %
                \node[draw,line width=5pt,text width=0.6\linewidth,align=center,rounded corners=0.5cm,inner sep=24] (solver) at ($(origin)+(0,-17)$) {\textbf{\texttt{El0ps}} \\
                Branch-and-Bound solver \\
                Tunable strategies \\
                Acceleration methods};
                %
                \draw[->,line width=5pt] (lossfunc.south) -- ($(lossfunc.south)+(0,-2)$);
                \draw[->,line width=5pt] (pertfunc.south) -- ($(pertfunc.south)+(0,-2)$);
            \end{tikzpicture}
        \end{figure}
    \end{block}
\end{column}

\begin{column}{\sepwid}\end{column}

\begin{column}{\twocolwid}    
    \begin{columns}[t,totalwidth=\twocolwid]
        \begin{column}{\onecolwid}\vspace{-.6in} 
            \begin{block}{Branch-and-Bound algorithms}
                \textbf{Principle}
                \begin{center}
                    ``Partition the feasible space into \emphone{regions} and \emphone{prune} those that cannot contain minimizers.''
                \end{center}
                ~\\

                \textbf{Regions construction}
                \begin{itemize}[label=$\bullet \ $,leftmargin=2em]
                    \item $\nodeSymb = (\setzero,\setone)$ disjoint subsets of $\{1,\dots,\pdim\}$
                    \item $\setzero$ entries fixed to zero
                    \item $\setone$ entries fixed to non-zero
                \end{itemize}
                \begin{equation*}
                    \textcolor{norange}{\node{\pset} = \kset{\pv \in \kR^{\pdim}}{\pv_{\setzero} = \0, \ \pv_{\setone} \neq \0}}
                \end{equation*}
                \begin{itemize}[label=$\bullet \ $,leftmargin=2em]
                    \item All regions explored: problem solved
                \end{itemize}
                ~\\

                \textbf{Feasible space exploration}
                \begin{figure}
                    \centering
                    \forestset{
    node/.style = {
        draw, 
        circle,
        align = center,
        font = \small,
        top color = white,
        bottom color = blue!25,
        line width=5pt,
    },
    branch label/.style={
        edge label = {
            node[midway,fill=white,font=\small,draw=black,text height=0.5em,scale=0.8]{#1}
        }
    },
    bnb/.style={
        branch label,
        for tree = {
            node,
            s sep'+=40mm,
            l sep'+=10mm,
            edge={->},
            edge+={line width=5pt},
        },
        before typesetting nodes={
            for tree={
                split option={content}{:}{content,branch label},
            },
        },
        where n children=0{
            tikz+={
                \draw [dashed,->,line width=5pt]  ([yshift=0pt, xshift=0pt].south east) -- ([yshift=-15pt, xshift=15pt].south east);
                \draw [dashed,->,line width=5pt]  ([yshift=0pt, xshift=0pt].south west) -- ([yshift=-15pt, xshift=-15pt].south west);
            }
        }{},
    },
}
\begin{forest}
    bnb,
    [
        \(\nodeSymb_{0}\)
            [\(\nodeSymb_{1}\):\({\idxentry_0 \rightarrow \setzero}\)
            [\(\nodeSymb_{3}\):\({\idxentry_1 \rightarrow \setzero}\)]
            [\(\nodeSymb_{4}\):\({\idxentry_1 \rightarrow \setone}\)]
    ][
        \(\nodeSymb_{2}\):\({\idxentry_0 \rightarrow \setone}\),
            [\(\nodeSymb_{5}\):\({\idxentry_2 \rightarrow \setzero}\)]
            [\(\nodeSymb_{6}\):\({\idxentry_2 \rightarrow \setone}\)]
    ]
    ]
\end{forest}
                \end{figure}

                \textbf{Processing nodes}
                \begin{itemize}[label=$\bullet \ $,leftmargin=2em]
                    \item Node problem
                \end{itemize}
                \begin{equation*}
                    \textstyle
                    \node{\pobj} = 
                    \min_{\textcolor{norange}{\pv \in \node{\pset}}} \lossfunc(\dic\pv) + \reg\norm{\pv}{0} + \pertfunc(\pv)
                \end{equation*}
                \begin{itemize}[label=$\bullet \ $,leftmargin=2em]
                    \item Test if region $\node{\pset}$ contains minimizers: $\node{\pobj} > \opt{\pobj}$
                    \item Practical \textcolor{norange}{pruning test} with bounds
                    \begin{equation*}
                        \node{\pobj} \geq \textcolor{norange}{\node{\LB{\pobj}} > \UB{\pobj}} \geq \opt{\pobj}
                    \end{equation*}
                    \item Upper bound $\UB{\pobj}$ on $\opt{\pobj}$
                    \begin{itemize}[label=$\rightarrow \ $,leftmargin=2em]
                        \item Evaluate objective at any point
                        \item Efficient heuristics
                    \end{itemize}
                \end{itemize}
            \end{block}
        \end{column}

        \begin{column}{\onecolwid}\vspace{-.6in}
            \begin{block}{Generic relaxations}
                \begin{alertblock}{Constructing lower-bounds}
                    \centering
                    \begin{tikzpicture}
                        \node (node-func) at (current page.north) {\textbf{Node problem reformulation}};
                        \node (node-func-eq) at ($(node-func.south)+(0,-1)$) {$\node{\pobj} = \textstyle\min_{\pv \in \kR^{\pdim}} \lossfunc(\dic\pv) + \textcolor{norange}{\node{\regfunc}}(\pv)$};
                        \node (node-func-eq2) at ($(node-func-eq.south)+(0,-3)$) {with $\node{\separable{\regfunc}{\idxentry}}(\pvi{\idxentry}) = \begin{cases}
                            \mathbf{I}(\pvi{\idxentry} = 0) &\text{if} \  \idxentry \in \setzero \\
                            \separable{\pertfunc}{\idxentry}(\pvi{\idxentry}) + \reg &\text{if} \  \idxentry \in \setone \\
                            \separable{\pertfunc}{\idxentry}(\pvi{\idxentry}) + \reg\norm{\pvi{\idxentry}}{0} &\text{otherwise}
                        \end{cases}$};
                        %
                        \node (primal-relax) at ($(node-func-eq2.south)+(0,-2)$) {\textbf{Primal relaxation}};
                        \node (primal-pb) at ($(primal-relax.south)+(0,-1)$) {$\textstyle\min_{\pv \in \kR^{\pdim}} \lossfunc(\dic\pv) + \textcolor{norange}{\biconj{(\node{\regfunc})}}(\pv)$};
                        %
                        \node (dual-relax) at ($(primal-pb.south)+(0,-2)$) {\textbf{Dual relaxation}};
                        \node (primal-pb) at ($(dual-relax.south)+(0,-1)$) {$\textstyle\max_{\dv \in \kR^{\ddim}} -\conj{\lossfunc}(-\dv) + \textcolor{norange}{\conj{(\node{\regfunc})}}(\transp{\dic}\dv)$};
                    \end{tikzpicture}
                \end{alertblock}

                \textbf{Closed-form expressions}
                \begin{itemize}[label=$\bullet \ $,leftmargin=2em]
                    \item 1D-study of $\regfunc(\pvi{}) = \pertfunc(\pvi{}) + \reg\norm{\pvi{}}{0}$
                    \item Parameters \textcolor{norange}{$(\pertslope,\pertlimit)$} easily computable from \textcolor{norange}{$\conj{\pertfunc}$}
                    \item Closed-form expressions
                \end{itemize}
                \begin{align*}
                    \biconj{\regfunc}(\pvi{}) &= 
                    \begin{cases}
                        \textcolor{norange}{\pertslope}\abs{\pvi{}} &\text{if} \ \abs{\pvi{}} \leq \textcolor{norange}{\pertlimit} \\
                        \pertfunc(\pvi{}) + \reg \ \, \, &\text{if} \ \abs{\pvi{}} > \textcolor{norange}{\pertlimit}
                    \end{cases} \\
                    \conj{\regfunc}(\dvi{}) &= 
                    \begin{cases}
                        0 &\text{if} \ \abs{\dvi{}} \leq \textcolor{norange}{\pertslope} \\
                        \conj{\pertfunc}(\dvi{}) - \reg &\text{if} \ \abs{\dvi{}} > \textcolor{norange}{\pertslope}
                    \end{cases}
                \end{align*}
                \begin{itemize}[label=$\bullet \ $,leftmargin=2em]
                    \item Simple evaluation of prox, subdiff, ...
                \end{itemize}
                \setlength{\fboxrule}{5pt}
                \fbox{
                \begin{minipage}{\textwidth}
                \begin{figure}
                    \centering
                    \begin{tikzpicture}
    \begin{scope}[xscale=6,yscale=6]
        \node (origin) at (0,0) {};
        \draw[line width=5pt,->] (-0.85,0) -- (0.85, 0);
        \draw[line width=5pt,->] (0,0) -- (0, 1);
        \draw[line width=5pt] (-0.03,0.4) -- (0.03,0.4);
        \node[above right] at (0,0.42) {\small{$\reg$}};
        \node[right] at (0.85,0) {$\pvi{}$};
        \node[above] at (0,1) {\small{$\pertfunc \equiv$ bound-cstr.}};
        %
        \draw[{Arc Barb[arc=130,reversed]}-,nblue,line width=5pt] (-0.5,0.4) plot[domain=0.03:0.5] (\x,0.4);
        \draw[-{Arc Barb[arc=130,reversed]},nblue,line width=5pt] (-0.5,0.4) plot[domain=-0.5:-0.03] (\x,0.4) node {};
        \draw[nblue,line width=5pt] (0.5,0.4) -- (0.5,0.85);
        \draw[nblue,line width=5pt] (-0.5,0.4) -- (-0.5,0.85);
        \draw[norange,line width=5pt] (0, 0) -- (0.515, 0.415);
        \draw[norange,line width=5pt] (0.51,0.4) -- (0.51,0.85);
        \draw[norange,line width=5pt] (0, 0) -- (-0.515, 0.415);
        \draw[norange,line width=5pt] (-0.51,0.4) -- (-0.51,0.85);
        \fill[nblue] (0,0) circle (0.03);
        %
        \draw[line width=5pt] (0.5,-0.03) -- (0.5,0.03);
        \node[below] at (0.5,0) {\scriptsize{$\pertlimit$}};
        \draw[line width=3pt,densely dashed] (0.5,0) -- (0.5,0.4);
        \draw[line width=3pt,densely dashed] (0, 0) -- (0.75, 0.6);
        \node[rotate=38,anchor=center] at (0.3,0.15) {\scriptsize{slope $\pertslope$}};
        %
        \fill[nblue] (0,0) circle (0.03);
        %
        \node at (0.45,0.92) {\small{\textcolor{nblue}{$\regfunc$}}};
        \node at (0.65,0.95) {\small{\textcolor{norange}{$\biconj{\regfunc}$}}};
    \end{scope}
    \begin{scope}[shift={(0.5\linewidth,0)},xscale=6,yscale=6]
        \node (origin) at (0,0) {};
        \draw[line width=5pt,->] (-0.85,0) -- (0.85, 0);
        \draw[line width=5pt,->] (0,0) -- (0, 1);
        \draw[line width=5pt] (-0.03,0.2) -- (0.03,0.2);
        \node[above right] at (0,0.2) {\small{$\reg$}};
        \node[right] at (0.85,0) {$\pvi{}$};
        \node[above] at (0,1) {\small{$\pertfunc \equiv$ $\ell_2$-norm}};
        %
        \draw[{Arc Barb[arc=130,reversed]}-,nblue,line width=5pt] (-0.5,0.2) plot[domain=0.03:0.8] (\x,0.2+\x^2);
        \draw[-{Arc Barb[arc=130,reversed]},nblue,line width=5pt] (-0.5,0.2) plot[domain=-0.8:-0.03] (\x,0.2-\x^2) node {};
        %
        \draw[norange,line width=5pt] (0, 0) -- (0.5, 0.43);
        \draw[norange,line width=5pt] (0.5, 0.43) plot[domain=0.5:0.8] (\x,0.2+\x^2-0.02) node {};
        \draw[norange,line width=5pt] (0, 0) -- (-0.5, 0.43);
        \draw[norange,line width=5pt] (-0.5, 0.43) plot[domain=-0.8:-0.5] (\x,0.2-\x^2-0.02) node {};
        \fill[nblue] (0,0) circle (0.03);
        %
        \draw[line width=3pt,densely dashed] (0, 0) -- (1, 0.86);
        \node[below] at (0.5,0) {\scriptsize{$\pertlimit$}};
        \draw[line width=5pt] (0.5,-0.03) -- (0.5,0.03);
        \draw[line width=3pt,densely dashed] (0.5,0) -- (0.5,0.42);
        \node[rotate=40,anchor=center] at (0.3,0.15) {\scriptsize{slope $\pertslope$}};
        %
        \fill[nblue] (0,0) circle (0.03);
        %
        \node at (0.7,0.97) {\small{\textcolor{nblue}{$\regfunc$}}};
        \node at (0.95,1) {\small{\textcolor{norange}{$\biconj{\regfunc}$}}};
    \end{scope}
\end{tikzpicture}
                \end{figure}
                \begin{figure}
                    \centering
                    \begin{tikzpicture}
    \begin{scope}[xscale=6,yscale=6]
        \draw[line width=5pt,->] (-0.85,0) -- (0.85, 0);
        \draw[line width=5pt,->] (0,0) -- (0, 1);
        \node[right] at (0.85,0) {$\dvi{}$};
        % \node[above] at (0,1) {\small{$\pertfunc \equiv$ bound-cstr.}};
        %
        \draw[nblue,line width=5pt] (-0.75,0.8) -- (0,0);
        \draw[nblue,line width=5pt] (0,0) -- (0.75,0.8);
        \draw[norange,line width=5pt] (-0.4,0) -- (0.4,0);
        \draw[norange,line width=5pt] (-0.39,-0.01) -- (-0.77,0.4);
        \draw[norange,line width=5pt] (0.39,-0.01) -- (0.77,0.4);
        %
        \draw[line width=3pt,densely dashed] (0.4,0) -- (0.4, 0.4);
        \draw[line width=3pt,densely dashed] (0,0.4) -- (0.4, 0.4);
        \draw[line width=5pt] (0.4,-0.03) -- (0.4,0.03);
        \node[below] at (0.4,0) {\scriptsize{$\pertslope$}};
        \draw[line width=5pt] (-0.03,0.4) -- (0.03,0.4);
        \node[above right] at (0,0.4) {\small{$\reg$}};
        %
        \node at (0.9,0.8) {\small{\textcolor{nblue}{$\conj{\pertfunc}$}}};
        \node at (0.9,0.4) {\small{\textcolor{norange}{$\conj{\regfunc}$}}};
    \end{scope}
    \begin{scope}[shift={(0.5\linewidth,0)},xscale=6,yscale=6]
        \draw[line width=5pt,->] (-0.85,0) -- (0.85, 0);
        \draw[line width=5pt,->] (0,0) -- (0, 1);
        \node[right] at (0.85,0) {$\dvi{}$};
        % \node[above] at (0,1) {\small{$\pertfunc \equiv$ $\ell_2$-norm}};
        %
        \draw[nblue,line width=5pt] (-0.5,0.2) plot[domain=-0.8:0] (\x,-1.5*\x^2);
        \draw[nblue,line width=5pt] (-0.5,0.2) plot[domain=0:0.8] (\x,1.5*\x^2);
        %
        \draw[norange,line width=5pt] (-0.5,0.2) plot[domain=-0.82:-0.53] (\x,-0.28-1*\x^2);
        \draw[norange,line width=5pt] (-0.5,0.2) plot[domain=0.53:0.82] (\x,-0.28+1*\x^2);
        \draw[norange,line width=5pt] (-0.53,0) -- (0.53,0);
        %
        \draw[line width=3pt,densely dashed] (0.53,0) -- (0.53,0.4);
        \draw[line width=3pt,densely dashed] (0,0.4) -- (0.53,0.4);
        \draw[line width=5pt] (0.53,-0.03) -- (0.53,0.03);
        \node[below] at (0.53,0) {\scriptsize{$\pertslope$}};
        \draw[line width=5pt] (-0.03,0.4) -- (0.03,0.4);
        \node[above right] at (0,0.4) {\small{$\reg$}};
        %
        \node at (0.9,0.9) {\small{\textcolor{nblue}{$\conj{\pertfunc}$}}};
        \node at (0.9,0.5) {\small{\textcolor{norange}{$\conj{\regfunc}$}}};
    \end{scope}
\end{tikzpicture}
                \end{figure}
                \end{minipage}
                }
            \end{block}
        \end{column}
    \end{columns}
\end{column}

\begin{column}{\sepwid}\end{column}

\begin{column}{\onecolwid}
    \begin{block}{Numerical results}
        \textbf{Standard instances}

        \begin{itemize}[label=$\bullet \ $,leftmargin=2em]
            \item $\lossfunc$: least-squares
            \item $\pertfunc$: bound-cstr. or $\ell_2$-norm (left/right)
            \item $\reg$: tuned statistically
        \end{itemize}
        \begin{figure}
            \centering
            \begin{tikzpicture}
    \begin{groupplot}[
        group style     = {
            group size      = 2 by 1,
            xticklabels at  = edge bottom,
            yticklabels at  = edge left,
            vertical sep    = 10pt,
            horizontal sep  = 70pt,
        },
        height          = 0.35\linewidth,
        width           = 0.4\linewidth,
        ymin            = -0.05,
        ymax            = 1.05,
        xmode           = log,
        ytick           = {0,0.25,0.5,0.75,1},
        yticklabels     = {0\%,25\%,50\%,75\%,100\%},
        xtick           = {0.01,0.1,1,10,100,1000,10000},
        xticklabels     = {$10^{-2}$, , ,$10^{1}$, , ,$10^{4}$},
        ytick pos       = left,
        xtick pos       = bottom,
        grid            = major,
    ]
    
        % ------------------
        % FIRST ROW
        % ------------------

        \nextgroupplot[
            xlabel          = Time,
            ylabel          = Prop. solved,
            cycle list name = solver_std_cycle_list,
        ]
        \foreach \solver in {Cplex,Scip,Mosek,L0Bnb,SBnb,El0ps}{
            \addplot table[
                x       = times, 
                y       = \solver,
                col sep = comma,
            ] {dat/perfprofiles_Leastsquares_Bigm.csv};
            \label{plots:perfprofiles_\solver}
        };
        \coordinate (top) at (rel axis cs:0,1);

        \nextgroupplot[
            xlabel          = Time,
            cycle list name = solver_std_cycle_list,
        ]
        \foreach \solver in {Cplex,Scip,Mosek,L0Bnb,SBnb,El0ps}{
            \addplot table[
                x       = times, 
                y       = \solver,
                col sep = comma,
            ] {dat/perfprofiles_Leastsquares_BigmL2norm.csv};
        };
        \coordinate (bot) at (rel axis cs:1,0);
    \end{groupplot}
    \path (top|-current bounding box.north)--
    coordinate(legendpos)
    (bot|-current bounding box.north);
    \matrix[
        matrix of nodes,
        anchor=north,
        draw,
        inner sep=0.2em,
        every node/.style={anchor=base west, font=\small},
    ] at([yshift=6ex,xshift=-3ex]legendpos)
    {
        \ref{plots:perfprofiles_Cplex} & \texttt{Cplex} & [10pt]
        \ref{plots:perfprofiles_Scip} & \texttt{Scip} & [10pt]
        \ref{plots:perfprofiles_Mosek} & \texttt{Mosek} & [10pt] \\ 
        \ref{plots:perfprofiles_L0Bnb} & \texttt{L0bnb} & [10pt]
        \ref{plots:perfprofiles_SBnb} & \texttt{Sbnb} & [10pt]
        \ref{plots:perfprofiles_El0ps} & \texttt{El0ps} \\
    };
\end{tikzpicture}
        \end{figure}
        \vspace*{-1.5cm}
            
        \textbf{New application opportunities}
        \begin{itemize}[label=$\bullet \ $,leftmargin=2em]
            \item $\lossfunc$: logistic
            \item $\pertfunc$: bound-cstr. + $\ell_1$- or $\ell_2$-norm (left/right)
            \item $\reg$: tuned statistically
        \end{itemize}
        \begin{figure}
            \centering
            \begin{tikzpicture}
    \begin{groupplot}[
        group style     = {
            group size      = 2 by 1,
            xticklabels at  = edge bottom,
            yticklabels at  = edge left,
            vertical sep    = 10pt,
            horizontal sep  = 70pt,
        },
        height          = 0.35\linewidth,
        width           = 0.4\linewidth,
        ymin            = -0.05,
        ymax            = 1.05,
        xmode           = log,
        ytick           = {0,0.25,0.5,0.75,1},
        yticklabels     = {0\%,25\%,50\%,75\%,100\%},
        xtick           = {0.01,0.1,1,10,100,1000,10000},
        xticklabels     = {$10^{-2}$, , ,$10^{1}$, , ,$10^{4}$},
        ytick pos       = left,
        xtick pos       = bottom,
        grid            = major,
    ]
    
        % ------------------
        % FIRST ROW
        % ------------------

        \nextgroupplot[
            xlabel          = Time,
            ylabel          = Prop. solved,
            cycle list name = solver_new_cycle_list,
        ]
        \foreach \solver in {Mosek,El0ps}{
            \addplot table[
                x       = times, 
                y       = \solver,
                col sep = comma,
            ] {dat/perfprofiles_Logistic_BigmL1norm.csv};
            \label{plots:perfprofiles_\solver}
        };
        \coordinate (top) at (rel axis cs:0,1);

        \nextgroupplot[
            xlabel          = Time,
            cycle list name = solver_new_cycle_list,
        ]
        \foreach \solver in {Mosek,El0ps}{
            \addplot table[
                x       = times, 
                y       = \solver,
                col sep = comma,
            ] {dat/perfprofiles_Logistic_BigmL2norm.csv};
        };
        \coordinate (bot) at (rel axis cs:1,0);
    \end{groupplot}
    \path (top|-current bounding box.north)--
    coordinate(legendpos)
    (bot|-current bounding box.north);
    \matrix[
        matrix of nodes,
        anchor=north,
        draw,
        inner sep=0.2em,
        every node/.style={anchor=base west, font=\small},
    ] at([yshift=4ex,xshift=-3ex]legendpos)
    {
        \ref{plots:perfprofiles_Mosek} & \texttt{Mosek} & [10pt]
        \ref{plots:perfprofiles_El0ps} & \texttt{El0ps} \\
    };
\end{tikzpicture}
        \end{figure}
        \vspace*{-1cm}
        \begin{center}
            \large\textcolor{ForestGreen}{\ding{51} \textbf{Fast}} \\
            \large\textcolor{ForestGreen}{\ding{51} \textbf{Flexible w.r.t $\lossfunc/\pertfunc$}}
        \end{center}
        ~\\
        \setbeamercolor{block alerted title}{fg=white,bg=norange}
        \setbeamercolor{block alerted body}{fg=black,bg=white}
        \begin{alertblock}{Take home message}
            \centering
            Opportunities to address new instances of $\ell_0$-problems efficiently.
        \end{alertblock}
    \end{block}
\end{column} 

\begin{column}{\sepwid}\end{column}

\end{columns}
\end{frame}

\end{document}
